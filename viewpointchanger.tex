% $Id: template.tex 11 2007-04-03 22:25:53Z jpeltier $

%\documentclass{vgtc}                          % final (conference style)
\documentclass[review]{vgtc}                 % review
%\documentclass[widereview]{vgtc}             % wide-spaced review
%\documentclass[preprint]{vgtc}               % preprint
%\documentclass[electronic]{vgtc}             % electronic version
\let\ifpdf\relax

%% Uncomment one of the lines above depending on where your paper is
%% in the conference process. ``review'' and ``widereview'' are for review
%% submission, ``preprint'' is for pre-publication, and the final version
%% doesn't use a specific qualifier. Further, ``electronic'' includes
%% hyperreferences for more convenient online viewing.

%% Please use one of the ``review'' options in combination with the
%% assigned online id (see below) ONLY if your paper uses a double blind
%% review process. Some conferences, like IEEE Vis and InfoVis, have NOT
%% in the past.

%% Please note that the use of figures other than the optional teaser is not permitted on the first page
%% of the journal version.  Figures should begin on the second page and be
%% in CMYK or Grey scale format, otherwise, colour shifting may occur
%% during the printing process.  Papers submitted with figures other than the optional teaser on the
%% first page will be refused.

%% These three lines bring in essential packages: ``mathptmx'' for Type 1
%% typefaces, ``graphicx'' for inclusion of EPS figures. and ``times''
%% for proper handling of the times font family.

\usepackage{mathptmx}
\usepackage{graphicx}
\usepackage{times}
\usepackage{enumerate}
\usepackage{color}
\usepackage{bm}
\usepackage{amsmath}
\usepackage{subfigure}

%% We encourage the use of mathptmx for consistent usage of times font
%% throughout the proceedings. However, if you encounter conflicts
%% with other math-related packages, you may want to disable it.

%% If you are submitting a paper to a conference for review with a double
%% blind reviewing process, please replace the value ``0'' below with your
%% OnlineID. Otherwise, you may safely leave it at ``0''.
\onlineid{2277}

%% We encourage the use of mathptmx for consistent usage of times font
%% throughout the proceedings. However, if you encounter conflicts
%% with other math-related packages, you may want to disable it.

%% This turns references into clickable hyperlinks.
%% If you are submitting a paper to a conference for review with a double
%% blind reviewing process, please replace the value ``0'' below with your
%% OnlineID. Otherwise, you may safely leave it at ``0''.
\onlineid{2277}
\newcommand{\note}[1]{\iffalse #1 \fi}
\newcommand{\col}[1]{{\color{blue}{#1}}}
\note{
\usepackage[bookmarks,backref=true,linkcolor=black]{hyperref} %,colorlinks
\hypersetup{
  pdfauthor = {},
  pdftitle = {},
  pdfsubject = {},
  pdfkeywords = {},
  colorlinks=true,
  linkcolor= black,
  citecolor= black,
  pageanchor=true,
  urlcolor = black,
  plainpages = false,
  linktocpage
}
\tolerance=1
\emergencystretch=\maxdimen
\hyphenpenalty=10000
\hbadness=10000
}

%% declare the category of your paper, only shown in review mode
\vgtccategory{Research}

%% allow for this line if you want the electronic option to work properly
\vgtcinsertpkg

%% In preprint mode you may define your own headline.
%\preprinttext{To appear in an IEEE VGTC sponsored conference.}

%% Paper title.

\title{Exploring High Dimensional Data Through Locally Enhanced Projections}

%% This is how authors are specified in the journal style

%% indicate IEEE Member or Student Member in form indicated below
%%\author{Roy G. Biv, Ed Grimley, \textit{Member, IEEE}, and Martha Stewart}
%%\authorfooter{
%% insert punctuation at end of each item
%%\item
 %%Roy G. Biv is with Starbucks Research. E-mail: roy.g.biv@aol.com.
%%\item
%% Ed Grimley is with Grimley Widgets, Inc.. E-mail: ed.grimley@aol.com.
%%\item
%% Martha Stewart is with Martha Stewart Enterprises at Microsoft
%% Research. E-mail: martha.stewart@marthastewart.com.
%%}

%other entries to be set up for journal
%% \shortauthortitle{Biv \MakeLowercase{\textit{et al.}}: Global Illumination for Fun and Profit}
%\shortauthortitle{Firstauthor \MakeLowercase{\textit{et al.}}: Paper Title}

%% Abstract section.
\abstract{Dimension reduced projection accommodates all data in a low-dimensional space to approximate the high-dimensional distribution. It makes a good overview, but can hardly meet the needs of local data analyses. On the one hand, layout distortions largely harm the perception of local data relationships, especially in linear projections. On the other hand, non-linear projections seek to preserve local neighborhoods (e.g. Isomap, t-SNE) but at the expense of losing dimensional contexts. In fact, a sole projection can't be enough for local analyses with varying targets. It will be better to alter the layout according to users' point of interest (POI) and analytic tasks. In this paper, we extend such an idea into an interactive exploration scheme, to help users customize a linear projection for better local data analyses. Specifically, we first allow users to define their POI data. Then regarding different analytic tasks, we recommend projections and subspaces where different features of the POI are enhanced. Moreover, we provide various means to help users discover, analyze, modify and compare multiple POIs. At last, we demonstrate the effectiveness of our method via case studies with real-world datasets.
} % end of abstract

%% Keywords that describe your work. Will show as 'Index Terms' in journal
%% please capitalize first letter and insert punctuation after last keyword
\keywords{Dimension-reduced projection, local data analysis, high-dimensional data, subspace analysis}

%% ACM Computing Classification System (CCS).
%% See <http://www.acm.org/class/1998/> for details.
%% The ``\CCScat'' command takes four arguments.
\note{
\CCScatlist{ % not used in journal version
 \CCScat{K.6.1}{Management of Computing and Information Systems}%
{Project and People Management}{Life Cycle};
 \CCScat{K.7.m}{The Computing Profession}{Miscellaneous}{Ethics}
}
}

%% Uncomment below to include a teaser figure.
\note{
  }
  \teaser{
 \centering
 \includegraphics[width=0.9\linewidth]{images/new_pipeline1.eps}
  \caption{We propose an interactive and exploratory scheme for local analysis in high-dimensional data. (a) We first present a global projection as the overview. (b) Then suggestions are made to help user find an interesting piece of local data. (c) User chooses some local data as the focus, and named a feature to be analyzed. (d) We make linear projections to enhance local features of the chosen data. Different features are defined for different analytic tasks. (e) A feature-related subspace is also revealed to support dimensional analysis. From the resulting projection and subspace, users can gain both structural and dimensional insights about the data focus. (f) Finally, the user can move on to a new focus, and store valuable findings in Focus List for further study. Projection Map is provided to help compare multiple focuses in the list based on their features.
}
  \label{fig:workflow}
  }

%% Uncomment below to disable the manuscript note
%\renewcommand{\manuscriptnotetxt}{}

%% Copyright space is enabled by default as required by guidelines.
%% It is disabled by the 'review' option or via the following command:
% \nocopyrightspace

%%%%%%%%%%%%%%%%%%%%%%%%%%%%%%%%%%%%%%%%%%%%%%%%%%%%%%%%%%%%%%%%
%%%%%%%%%%%%%%%%%%%%%% START OF THE PAPER %%%%%%%%%%%%%%%%%%%%%%
%%%%%%%%%%%%%%%%%%%%%%%%%%%%%%%%%%%%%%%%%%%%%%%%%%%%%%%%%%%%%%%%%

\begin{document}

%% The ``\maketitle'' command must be the first command after the
%% ``\begin{document}'' command. It prepares and prints the title block.

%% the only exception to this rule is the \firstsection command

%% \section{Introduction} %for journal use above \firstsection{..} instead
\firstsection{Introduction}
\maketitle
Dimension-reduced projection is widely used for high-dimensional data analysis. It seeks to approximate the original distribution in a low-dimensional space. Such approximation is often globally optimized to make a good overview of the data. But due to approximation errors, data relationships will inevitably be distorted. The distortions are hard to ignore when it comes to local data. They largely harm the perception of local structures, yet are often transparent to users. Even if distortions are shown in the projection~\cite{DBLP:journals/tvcg/StahnkeDMT16}~\cite{DBLP:journals/ijon/Aupetit07}, users have no means to control their distribution. In general, globally optimized projections make good overviews, but are not suitable for local data analysis.
%change citation here.

One way to alleviate this problem, is to observe the data in a different perspective. Some previous works~\cite{DBLP:journals/cgf/JeongZFRC09}~\cite{DBLP:journals/tvcg/NamM13}~\cite{DBLP:journals/tvcg/LehmannT13} allow users to change the projection by adjusting dimension weights. It helps to clarify detailed structures and find out more useful information. To help build a targeted analysis, featured clusters are often provided beforehand~\cite{DBLP:journals/tvcg/NamM13}~\cite{DBLP:journals/cgf/LiuWTBP15}. However, users know nothing about the given clusters. They have to solve their puzzles by manually searching the enormous data space. It's a blinded and exhausting process, where users have no clue how to steer the dimensions to get a better view. Even if interesting projections are found, it's hard to explain or assess them without distortion information.

Compared to dimension-based exploration, feature-based mining techniques are more efficient in revealing informative projections. Projection pursuit~\cite{DBLP:journals/tc/FriedmanT74} searches for projections to optimize predefined indices. The rank-by-feature framework~\cite{DBLP:journals/ivs/SeoS05} ranks a series of projections by their interestingness. In more recent works, users are involved to specify desired features~\cite{DBLP:journals/tvcg/JohanssonJ09} and relationships~\cite{DBLP:journals/tvcg/HuBMHNL13}~\cite{DBLP:journals/tvcg/Gleicher13}. These methods, though effective, largely depend on predefined metrics or users' prior knowledge. It makes them unsuitable for an interactive exploration starting from scratch. Besides, little attention had been paid to facilitate local data analysis.

In this work, we propose an interactive exploration method, that promotes consecutive local data analyses in locally optimized projections.

\note{
Nevertheless, local analysis is not only necessary, but also an efficient way to explore high-dimensional data. On one hand, a sole overview cannot show all aspects.\note{especially when the dimensionality is high.} After a quick glance, users tend to pick up some local structure (e.g. clusters, outliers) and go into details. It helps them fully understand the data and discover more useful information. On the other hand, featured local data are good breakthrough points for an efficient exploration. Traditional methods~\cite{DBLP:journals/tvcg/NamM13}~\cite{DBLP:journals/tvcg/LehmannT13} allow users to change the projection by adjusting dimensions. But such exploration is blind and exhausting, since the data space is huge while users have no clue where to go. Compared to dimensions, data features are easier to perceive and explain. That's why dimension-based explorations often provide featured clusters as start points~\cite{DBLP:journals/tvcg/NamM13}~\cite{DBLP:journals/cgf/LiuWTBP15}. It's also the spirit behind feature-based projection mining techniques, such as projection pursuit~\cite{DBLP:journals/tc/FriedmanT74} and the rank-by-feature framework~\cite{DBLP:journals/ivs/SeoS05}.
}
\section{Related Work}
\label{section:relatedwork}
Out method facilitates local data analysis in high-dimensional data. It adopts projection pursuit to support data exploration, as opposed to dimension-driven exploration methods. We'll briefly introduce related works in the relevant fields.

\subsection{Data Locality Analysis in Projections}
Data locality has been extensively studied in high-dimensional data researches. There are roughly two branches, focusing on different aspects.

The major branch focuses on new projections, aiming to preserve local data relationships. Lots of non-linear projections have been proposed for this purpose, such as Laplacian Eigenmaps (LE)~\cite{DBLP:journals/neco/BelkinN03}, Locally Linear Embedding (LLE)~\cite{roweis2000nonlinear}, Local Tangent Space Alignment (LTSA)~\cite{DBLP:journals/corr/cs-LG-0212008}, etc. These methods fit for data lying on a low-dimensional manifold (e.g. images of faces). But the semantics of dimensions are weakened or lost. Users cannot know in which dimensions the data are similar or different. Hence, these methods are not suitable for general high-dimensional data with meaningful dimensions. In comparison, our method keeps all projections in a linear framework. It helps to explain data relationships in the context of dimensions.

Another branch aims to reveal distortions in a projection. Martins et al.~\cite{DBLP:journals/cg/MartinsCMT14} examined distortions in different types of projections. They used color mapping to indicate distortion levels, and helped find real neighborhoods with automatic algorithms. Liu et al. took a step~\cite{DBLP:journals/cgf/LiuWBP14} further by analyzing data structures based on distortions. But none of them provide means to correct the distorted relationships. Stahnke et al.~\cite{DBLP:journals/tvcg/StahnkeDMT16} corrected the layout by directly changing point distances. However, such method is too straightforward to support a more in-depth local analysis. We also offer to show and correct distortions in this work. But the correction here is achieved by changing the whole projection. It can reduce distortions for a small neighborhood rather than a single datum.

\subsection{Projection Assisted Data Exploration}
Dimension reduced projections are often used to explore high-dimensional data. They are intuitive overviews, but hard to be changed interactively. Jeong et al.~\cite{DBLP:journals/cgf/JeongZFRC09} allow users to change a projection by updating dimension weights in the PCA algorithm. Nam et al.~\cite{DBLP:journals/tvcg/NamM13} took a step further. They let users directly decide both axes of the projection. More recently, Lehmann et al.~\cite{DBLP:journals/tvcg/LehmannT13} proposed a more intuitive interaction. Users can alter the dimension axes while maintaining an orthogonal mapping. These methods are indeed effective. But users have to go through an exhausting trial-and-error process, since they cannot foresee the effects of changing parameters. Our method allow users to manipulate data and relationships, rather than dimension weights. It's more intuitive and more efficient. Users don't have to search by themselves, but can still get the most insightful views.

In a projection assisted exploration, subspace clusters are often provided beforehand~\cite{DBLP:conf/ieeevast/TatuMFBSSK12}~\cite{DBLP:journals/tvcg/NamM13}~\cite{DBLP:journals/cgf/LiuWTBP15}. In some other methods~\cite{DBLP:journals/tois/ChenL06}~\cite{DBLP:conf/ieeevast/NamHMZI07}~\cite{DBLP:conf/ieeevast/TatuMFBSSK12}, users are able to participate in the automatic clustering process. But in either way, users don't fully understand the given clusters or subspaces. It's hard for them to modify the results, let alone discovering more hidden clusters. Yuan et al.~\cite{DBLP:journals/tvcg/YuanRWG13} proposed a hierarchical subspace exploration, which is the most relevant to our method. They allow users to analyze a local subset in different subspaces. It helps to discover hidden local clusters. However, they did not provide any guidance for subspace selection. Besides, the lack of context makes it hard to comprehend and modify the local data.
\note{
	\textbf{Subspace Cluster Estimation}
	\note{~\cite{DBLP:journals/tsp/CarterRH10}}%Intrinsic Dimension
	~\cite{DBLP:conf/ieeevast/Kandogan12}%Just-in Time
	~\cite{DBLP:journals/tvcg/YuanRWG13}%Subspace Cluster
}

\subsection{Feature Driven Projection Selection}
Projection pursuit~\cite{DBLP:journals/tc/FriedmanT74}~\cite{cook1995grand} is a well-known technique for finding interesting projections. It generates a series of projections to optimize a certain index. Gleicher et al.~\cite{DBLP:journals/tvcg/Gleicher13} used machine learning to train compositive dimensions for classification. Choo et al.~\cite{DBLP:conf/ieeevast/ChooLKP10} made the process interactive by involving users in a semi-supervised Linear Discriminant Analysis (LDA). In both works, user-defined classes are imported as the pursuit index. Apart from the classes, user-defined layouts can also act as the target~\cite{DBLP:journals/tvcg/JoiaCCPN11}~\cite{DBLP:conf/ieeevast/BrownLBC12}~\cite{DBLP:journals/tvcg/HuBMHNL13}. These works require the user to have solid prior knowledge. However, it's often not the case in a data exploration.

The rank-by-feature framework~\cite{DBLP:journals/ivs/SeoS05} is a variant of projection pursuit. It ranks existing projections according to feature strengths. Various kinds of metrics~\cite{DBLP:conf/ieeevast/AlbuquerqueEM11} are defined to measure different features, including class separation~\cite{DBLP:journals/cgf/SipsNLH09}~\cite{DBLP:journals/cgf/SedlmairTMT12}, clustering / outliering~\cite{DBLP:conf/ieeevast/TatuAESTMK09}~\cite{DBLP:journals/tvcg/JohanssonJ09}, and more complex topological properties~\cite{DBLP:conf/infovis/WilkinsonAG05}. They are helpful for analyzing a large group of scatterplots~\cite{DBLP:conf/apvis/NhonW14}~\cite{DBLP:conf/ieeevast/AnandWN12}. But it's hard to use complex measurements as criteria in projection pursuit. The time spent to find and score a projection will be unbearable in an interactive exploration. In our method, we adopt the strategy of projection pursuit, using simple criteria to describe local data relationships. The simple criteria are not only more efficient, but also easier to interpret.

\note{
~\cite{DBLP:conf/infovis/WilkinsonAG05}%Scagnostics
~\cite{DBLP:conf/ieeevast/AlbuquerqueEM11}%quality measures
~\cite{DBLP:conf/ieeevast/TatuAESTMK09}%combining automatic analysis
~\cite{DBLP:journals/cgf/SipsNLH09}%class consistency
~\cite{DBLP:journals/cgf/SedlmairTMT12}%class separation
~\cite{DBLP:journals/tvcg/JohanssonJ09}%combination of metrics
~\cite{DBLP:conf/apvis/NhonW14}%scagexplorer
~\cite{DBLP:conf/ieeevast/AnandWN12}%random projection

\textbf{Projection Pursuit for Classification}
~\cite{DBLP:conf/ieeevast/ChooLKP10}%iVisClassifier

\textbf{Targeted Projection Pursuit}
}
\section{Overview}
\label{section:overview}

\note{
\col{Outline for overview:
\begin{enumerate}[(1)]
\item Reducing local distortion
\item Enhancing local relationships
\item Workflow
\end{enumerate}
}

In this work, we propose local featured projections, to facilitate the analyses of user-defined focuses, to facilitate high-dimensional local data analysis. These projections

In this work, we propose to steer high-dimensional data exploration via local analyses in linear projections. To facilitate the analysis, we help to reduce local distortions, and enhance local relationships. In this section, we first clarify the concept of distortion reduction and relationship enhancement. Then we give an overview of the interactive exploration process supported by the locally optimized projections.

\subsection{Local Distortion Reduction}
Distortion usually refers to the gap between original data distances and the projected distances. We call it the \emph{distance distortion}. Our approach helps to reduce such distortions locally for a more faith perception. However, it may not be enough to support data feature analyses. There have been lots of dimension reduction techniques~\cite{jolliffe2002principal}~\cite{borg2005modern}~\cite{tenenbaum2000global}~\cite{roweis2000nonlinear}, aiming to reduce distance distortions globally. But a more recent research~\cite{DBLP:journals/tvcg/EtemadpourMPMOL15} has shown that, those projections cannot guarantee a good performance for certain analytic tasks. The main cause, in our opinion, is the existence of relationship distortions.

By 'relationship', we refer to a relative concept of distance, i.e. 'close' or 'far'. In the high-dimensional space, relationship should be defined in a certain data range and dimensional subspace. Assume that there are four data items distributed in a two-dimensional plane, as shown in Figure. \col{(Figure to be added.)} When talking about data A, B and C, we can say that C is far away from B (compared to A). But when talking about data B, C and D, C seems close to B (compared to D). On the other hand, C is closer to B than A in dimension X, while the opposite happens in dimension Y. When combing all data and dimensions, weaker relationships give way to the stronger ones. The weak ones (e.g. C is closer to B than A) can no longer be perceived. Even the strong ones are not as obvious as in the original context. We call it the \emph{relationship distortion}. The situation is alike in more complex real-world datasets. Integrated measurements cannot reflect local relationships precisely. That's why approximating the overall distances cannot guarantee enhanced data features.

In contrast, reducing distortion of the interested relationship can support a more targeted analysis. For example, when we talk about two objects being similar, we tend to ignore their differences temporally and vice versa. We believe that distortion reduction should be built upon certain analytic targets, namely the local data and relationships. It is the key to revealing hidden local features. That's the reason we provide two different types of distortion reduction, for different purposes. In fact, we will prove that distance distortion is a special case of relationship distortion. Both goals can actually be achieved in the same framework.\note{ Specifically, we allow users to focus on a data subset to accommodate relationships in different ranges. In addition, projection pursuit is applied to enhance relationships in different subspaces.} More details will be introduced in Section~\ref{section:method}.

\subsection{Workflow}
}

In this work, we aim to facilitate high-dimensional local data analysis in linear projections. Generally, we let the user specify his focus in any projection, and customize the projection for a better analysis regarding the focus. We propose an interactive exploration scheme, with various kinds of techniques to help the user find, analyze, modify, and compare multiple focuses. To be specific, our method supports a four-step exploration process (Figure~\ref{fig:workflow}):
\note{
Our method promotes an efficient high-dimensional data exploration, by facilitating distortion-free local analysis. It not only helps to maintain a faithful perception, but aims to enhance local features for better information mining. To be specific, the exploration contains four steps (see Figure~\ref{fig:workflow}):
}
\begin{enumerate}[Step. 1:]
 \item \textbf{Focus Search}: First, we present a global projection as the overview (Figure~\ref{fig:workflow}(a)). In the given projection, we help the user find an interesting piece of local data. We assume a datum interesting if its distances to others has been distorted. We consider a group of data interesting if they appear as a cluster. Hence, we display distance distortions and projection clusters respectively, to suggest an interesting datum or a potential cluster (Figure~\ref{fig:workflow}(b)). The data chosen by the user, either a point or a group, is called a focus.
 \item \textbf{Enhancing Features of the Focus}: After some focus is chosen, we customize the projection to enhance its features. By features, we refer to three kinds of featured relationships we find most informative in the local analysis (Figure~\ref{fig:workflow}(d)). Different relationships serve different analytic purposes. After the feature enhancement, we reveal dimensions that are most related to the feature. Based on that, we suggest a subspace that can help interpret causes of the local relationships.
 \item \textbf{Focus Modification}: When the focus is a data group chosen in the projection, it could be a false cluster or missing some important pieces. The focus-enhanced projections provide insights from multiple aspects. They may reveal truths that could not be found in the previous projection, like sub-clusters, outliers, etc. We provide means to help shape the focus into a more consistent and complete group. Whenever the focus is updated, the focus-enhanced projections will also be updated. The process returns to Step~2. This loop goes on until the user fully understands the data in focus, or acquires a featured cluster. After that, he can save the results and transfer his attention to another focus, returning to Step~1.
 \item \textbf{Focus Comparison}: When some valuable local data is found, the user can store it in the focus list for a further analysis (Figure~\ref{fig:workflow}(f)). We also provide a 'Projection Map', which contains projections of all focuses in the list. The map helps to compare different focuses, and navigate the high-dimensional exploration.
\end{enumerate}
In the exploration process, users never have to worry about configuring the projections. They are able to transfer seamlessly between different local data. The generated layouts not only show hidden local relationships, but disclose their causes in the context of dimensions.  In the next section, we'll introduce in detail how we support the above exploration in each step.
\section{High-dimensional Local Data Analysis in Locally Enhanced Projections}
As described in the overview, the proposed method supports a four-step exploration. In this section, we'll elaborate details of our method in each step of the exploration process.
\label{section:method}
\subsection{Discovering Interesting Local Focus}
Shneiderman has suggested in his information seeking mantra~\cite{DBLP:conf/vl/Shneiderman96}: "Overview first, then detail on demand". Following the suggestion, we first provide the PCA projection as an overview of the data. Then we help the user find an interesting subset as the focus of subsequent local analysis.

In a projection, there are two situations where some local data is considered interesting. The first case is about distance distortion. Incorrect distances result in false neighborhoods. Closely distributed data in the projection may be far away in the original space and vice versa. Data involved in a distorted local area is regarded informative in the projection. It's also the basic idea in previous works concerning about data locality~\cite{DBLP:journals/cg/MartinsCMT14}~\cite{DBLP:journals/tvcg/StahnkeDMT16}. But such analysis only focuses on each datum at a time. It's hard to describe a group of data in this context. That's why we consider the second type, where the data is involved in some featured relationships, like being an outlier or a cluster. The relationship may have been weakened (e.g. a false cluster), but it's still strong enough to appear in the current projection. Hence, it makes a reasonable focus for a further study. Besides, it's suitable to describe a group of data in the context of relationships, rather than distance errors.

To put it simply, distortion analysis focuses more on the neighborhood of a single datum. Relationship analysis promotes the study of a data group. Regarding the two cases, we adopt different means to help the user find an interesting local focus.

\ifx
\begin{figure}[htbp]
\centering
\subfigure[Focus point Suggestion]{
	\includegraphics[width=0.48\linewidth]{images/suggestion_point.eps}
}
\subfigure[Focus Group Suggestion]{
	\includegraphics[width=0.46\linewidth]{images/suggestion_group.eps}
}
  \caption{Focus Suggestion: We provide}
\label{fig:suggestions}
  \end{figure}
  \else
\begin{figure}[htbp]
\centering
\includegraphics[width=1\linewidth]{images/suggestion.eps}
  \caption{Focus Suggestion: We help the user find an interesting local focus, by making two kinds of suggestions. (a) The focus point suggestion indicates projection distortions. If a datum has large distortions, it will be large in size with a color different from its neighbors. (b) The focus group suggestion simply shows clusters in the projection. We assume a group of data interesting if they appear as a projection cluster.}
\label{fig:suggestions}
\end{figure}
  \fi

\subsubsection{Focus Point Suggestion Based on Distance Distortion}
For any given projection, we consider a datum interesting if its distances to other data have been severely distorted. To measure the distortion, we accumulate distance errors for each datum in the projection:
\begin{equation}
Error(\mathbf{x}_{i}^{\prime}) = \sum\limits_{j=1}^{n}(Dist(\mathbf{x}_{i}, \mathbf{x}_{j}) - Dist(\mathbf{x}_{i}^{\prime}, \mathbf{x}_{j}^{\prime}))^{2}, i = 1,2,\cdots n
\end{equation}
Here the $\mathbf{x}_{i}$ and $\mathbf{x}_{i}^{\prime}$ represents the original data and the projected data respectively. Distance is measured by the Euclidean distance metric, taking into account all dimensions. We use point size to encode the accumulated distortion of each datum, as shown in Figure~\ref{fig:suggestions}(a).

On the other hand, we provide interactive hints to reveal the real distances. The approach is similar to that used in~\cite{DBLP:journals/tvcg/StahnkeDMT16}, but uses a different metaphor. When user hovers on the projection, we construct a so-called 'high-dimensional lantern' using interpolation. Assume that the hovered position corresponds to a two-dimensional datum $\mathbf{p}^{\prime}$, we interpolate its high-dimensional counterpart as follows:
\begin{equation}
\mathbf{p} = \sum\limits_{i=1}^{n}\mathbf{w}_{i}\cdot\mathbf{x}_{i} =  \sum\limits_{i=1}^{n} \left (\frac{Dist(\mathbf{x}_{i}^{\prime}, \mathbf{p}^{\prime})^{-1}}{\sum\limits_{j=1}^{n}Dist(\mathbf{x}_{j}^{\prime}, \mathbf{p}^{\prime})^{-1}}\right )\cdot\mathbf{x}_{i}
\end{equation}
The interpolation weight $\mathbf{w}_{i}$ of data $\mathbf{x}_{i}$ depends on its distance to the hovered spot in the projection. Closer data get larger weights. When user hovers right on $\mathbf{x}_{i}^{\prime}$, $\mathbf{w}_{i}$ equals $1$ while all the other weights get $0$. The result equals to the original data: $\mathbf{p} = \mathbf{x}_{i}$.

By the interpolation, we aims to infer what kind of data is desired by the user. Then this desired point acts as a high-dimensional lantern, shedding lights on all the other data to indicate it distances to them. With the lighting metaphor, we encode distance information using the saturation tunnel in HSL color space:
\begin{equation}
\begin{split}
Saturation(\mathbf{x}_{i}^{\prime}) &= \max{\{(\alpha D_{i}^{2} + \beta D_{i} + \gamma)^{-1}, 1\}},\\
D_{i} &= Dist(\mathbf{x}_{i}, \mathbf{p}), \quad i = 1,2,\cdots n
\end{split}
\end{equation}
The data gets high saturation, if it's close to the interpolated point in the original space. The parameters $\alpha,\ \beta$ and $\gamma$ come from the inverse-square law of the lighting model. Empirical values are chosen to accommodate most datasets. Users can hover on any datum to check its neighborhood. If the neighborhood has been distorted, its lights will not be able to illuminate its neighbors in the projection. In contrast, the real neighbors will appear in far away places. Figure~\ref{fig:suggestions}(a) demonstrates a case, where two neighboring points cannot affect each other. They are probably false neighbors due to distortion.

In summary, large data points are potentially interesting data with high distortion and inconsistent illumination. With the hints, user hovers around the projection like experiencing an adventure. He holds a lantern to explore unknown structures in the complex data space. Compared to~\cite{DBLP:journals/tvcg/StahnkeDMT16}, this method enables a more smooth and natural perception of distance information.

\subsubsection{Focus Group Suggestion Based on Projection Clusters}
Automatic clustering algorithms play an important role in previous works~\cite{DBLP:conf/ieeevast/NamHMZI07}~\cite{DBLP:journals/cgf/LeeKCSP12}~\cite{DBLP:journals/cgf/LiuWTBP15}. Users are either given the clustering results, or assisted in tuning parameters of the algorithm. However, it's not intuitive to drive the clustering by parameters, since the algorithm is often a black box to users. Besides, it's hard for users to understand causes and details about the clusters, let alone modifying them or discovering new ones.

In our method, we decide not to provide global clustering results. Instead, we suggest an interesting group of data by examining projection clusters. It is based on the fact that, no additional or prior knowledge should be assumed in a free exploration. Users choose their focuses based on what they perceive. We only reveal real structures of the chosen focus. Users can still take full control of the clustering process, after they get the local insights.

Lots of clustering algorithms can be applied to identify projection clusters~\cite{DBLP:conf/ieeevast/Kandogan12}. We adopt a variant of DBSCAN~\cite{zhou2012research} whose parameters are adaptive to the data. We choose DBSCAN because it can efficiently identify clusters in any shape. The self-adaptive parameters make it applicable to most datasets without the need of manual tuning. Refer to Figure~\ref{fig:suggestions}(b) for the effect of focus group suggestion. The potential clusters are shown as contours around the points. Users can choose any suggested cluster by simply clicking on it. To be clear, the suggestion only clarifies dominant relationships perceived by the user. It doesn't provide any extra information beyond the projection. If the user doesn't feel satisfied with the suggestion, he can choose his own focus by brushing the data.

\subsection{Featured Projections of the Focus}
We call a chosen datum the focus point, and call a chosen group the focus group. After some focus is chosen, we generate projections to either reduce its distortion, or enhance its local relationships.

\subsubsection{Distortion Reduced Projection}
For a focus point, we seek a projection to reduce its accumulated distance distortions. Let $\mathbf{P}$ be the focus point, we aim to solve the following optimization problem:
\begin{equation}
\min Error(\mathbf{P}) = \min_{\mathbf{A}}  \sum\limits_{i=1}^{n}(Dist(\mathbf{P}, \mathbf{x}_{i}) - Dist(\mathbf{PA}, \mathbf{x}_{i}^{\prime}))^{2}
\end{equation}
Here the term $\mathbf{A}$ represents the projection matrix. The projected focus point is $\mathbf{P^{\prime}} = \mathbf{PA}$. It can be proved that, the problem is equivalent to the following form:
\begin{equation}
\label{equation:center-shiftedPCA}
\max_{\mathbf{A}}  \sum\limits_{i=1}^{n}Dist(\mathbf{PA}, \mathbf{x}_{i}^{\prime})^{2}
\end{equation}
Note that if we replace $\mathbf{P}$ with the center of data $\mathbf{\bar{x}}$, the optimization will directly lead to PCA projection. In other words, the optimization can be regarded as PCA with a shifted center (i.e. the chosen focus). This center can reach its lowest distortion level. In turn, PCA can be thought of a single-datum distortion reduction. It lowers the overall distortion level, by helping the average datum. Figure~\ref{fig:car}(c) demonstrates the actual effect of our center-shifted PCA. As a result, the other data have more accurate distances towards the focus. False neighbors will be pulled away, leaving a more authentic neighborhood around the focus. Compared to the simple distance correction used in~\cite{DBLP:journals/tvcg/StahnkeDMT16}, our method preserves all benefits of a linear projection, rather than mere point-wise distances. It provides rich dimensional context, and help maintain a consistent mental model of the data space. In Section~\ref{subsubsection:relationship_enhancement}, we will further interpret this projection in the context of data relationships.

\subsubsection{Featured Local Relationships}
For a focus group, we first examine what kinds of relationships are of interest in the local analysis. Since relationships are defined based on distances, we can take a look at the distance matrix. Given the group, the distance matrix of all data is divided into three parts (Figure~\ref{fig:local_relationships}(a)). The first part describes distances between group members. The second part is about distances between the group and the other data. The last part describes distances among the context data. Since the last part has nothing to do with the focus, we simply ignore it. For the remaining parts, we consider the chosen data to be either 'similar' or 'dissimilar'.

\begin{figure}[htbp]
\centering
\includegraphics[width=1\linewidth]{images/enhancement1.eps}
  \caption{Enhancing featured local relationships. With a certain focus group, the overall distance matrix can be divided into four sub-matrices~(a). We enhance local relationships (similarity or dissimilarity)~(b), by enlarging or reducing distances in different sub-matrices. Three kinds of projections are generated as the results (c).}
\label{fig:local_relationships}
  \end{figure}

By revealing the similarities among group members, we show users in which aspects the data are most similar. It helps to comprehend why these data gather into a cluster in the projection. Enhancing dissimilarities, on the other hand, tells about the major differences among group members. Moreover, if there are sub-clusters within the group, the differences among them will be more prominent. This can reveal hidden local relationships.  Similarities between the group and the others is not of interest, as far as we are concerned. In contrast, by enhancing the dissimilarities, we can show why the focus group is different from the others. The idea resembles that of Linear Discriminant Analysis (LDA), expect that we did not regard the other data as a same class. In summary, three types of relationships are found most informative in the local analysis. We call them intra-group similarity, intra-group dissimilarity and inter-group dissimilarity respectively (Figure~\ref{fig:local_relationships}(b)).

\subsubsection{Relationship Enhanced Projections}
\label{subsubsection:relationship_enhancement}
With the three types of relationships, we first translate them in the context of data distances. Then we adopt projection pursuit to find linear projections for the enhancement.

Enhancing similarities or dissimilarities, equals to decreasing or enlarging data distances in the projection. For a focus group $G$, we enhance the intra-group dissimilarities by:
\begin{equation}
\max \sum\limits_{\mathbf{x}_{i}^{\prime}, \mathbf{x}_{j}^{\prime} \in G} Dist(\mathbf{x}_{i}^{\prime}, \mathbf{x}_{j}^{\prime})^{2} = \max_{\mathbf{A}} \sum\limits_{\mathbf{x}_{i}, \mathbf{x}_{j} \in G} Dist(\mathbf{x}_{i}\mathbf{A}, \mathbf{x}_{j}\mathbf{A})^{2}
\end{equation}
For simplicity, we call this optimization the \textbf{Expand} metric, since the focus group will be expanded in the resulting projection. The metric actually leads to a local PCA projection. Likewise, we enhance the similarities by maximizing the same metric:
\begin{equation}
\max_{\mathbf{A}} \sum\limits_{\mathbf{x}_{i}, \mathbf{x}_{j} \in G} Dist(\mathbf{x}_{i}\mathbf{A}, \mathbf{x}_{j}\mathbf{A})^{2}
\end{equation}
We call it the \textbf{Compress} metric, as the opposite of Expand. At last, we enhance the inter-focus dissimilarities by enlarging distances between the group and the other data:
\begin{equation}
\label{equation:Separate}
\max_{\mathbf{A}} \sum\limits_{\mathbf{x}_{i} \in G} \sum\limits_{\mathbf{x}_{j} \in \bar{G}} Dist(\mathbf{x}_{i}\mathbf{A}, \mathbf{x}_{j}\mathbf{A})^{2}
\end{equation}
This one is called the \textbf{Separate} metric. Figure~\ref{fig:local_relationships}(c) illustrate projections of all three metrics.

In fact, we can see a focus point as a group containing only one datum. There will not be Compress or Expand projections without multiple group members. But the Separate metric simply degrades to the single-datum distortion reduction (compare equation~(\ref{equation:center-shiftedPCA}) and~(\ref{equation:Separate})). That is, a datum has the lowest distortions, when it's far away from the others in the projection. This enables us to combine all featured projections into the same framework.

\subsubsection{Subspace Suggestion}
All dimensions are considered when pursuing the featured projections. However, only a few of them truly contribute to the features. The redundant dimensions will interfere with the analysis. That's why we need to reveal a subspace where features are most prominent.

In a sense, projection pursuit is a process to identify the most featured dimensions. We can make reliable suggestions according to its results. To be specific, we choose a subspace based on the projection found in the original space. Then in the chosen subspace, we do optimizations again to get the final results.

Given a projection, we first calculate the weight of each dimension:
\begin{equation}
W(d_{i}) = \left \|  \mathbf{a}_{i}\right \|_{2}, i = 1,2,\cdots m
\end{equation}
Here, $\mathbf{a}_{i}$ is the projected unit vector of dimension $d_{i}$. Its squared length acts as the weight. Then we rank all dimensions according to their weights: $W(d_{1}^{*}) > W(d_{2}^{*}) > \cdots >W(d_{m}^{*})$. All weights sum to 1. We pick out those with large weights, until their sum exceeds a certain threshold:
\begin{equation}
\begin{split}
&Subspace = \{d_{i}^{*}| i = 1,2, \cdots L \},\\
&s.t.\ \sum\limits_{j=1}^{L} W(d_{j}^{*}) \leq R \ \text{and}\ \sum\limits_{j=1}^{L+1} W(d_{j}^{*}) > R
\end{split}
\end{equation}
In fact, it is the same strategy as rank-by-feature~\cite{DBLP:journals/ivs/SeoS05}, except that our feature scores are dimension weights of a featured projection. The sum of weights is called the \textbf{subspace score}. It indicates how strong the chosen subspace is related to the features. The threshold $R$ is 0.75 by default, cutting down at least $25\%$ redundant dimensions. Users are informed of the weights at any time. They can change $R$ to include or exclude dimensions. After the subspace is chosen, we get the final result via a second-time projection pursuit in that subspace. The refined projection will be easier to interpret with only the most featured dimensions. But users can always decide whether to run the suggestion, or simply accept the original results.

\note{related to rank-by-feature}

\subsection{Modifying the Focus}
For a focus point, a distortion reduced projection is the final step. But for a focus group, it still needs to be modified. The featured projections support this task, by revealing the local insights.

The Expand projection shows minor relationships hidden in the group. Sub-clusters and outliers can be found. It helps to trim the focus into a more consistent cluster. The Compress projection not only shows similar aspects within the group. There could be other data who resemble the focus in these aspects. They will be drawn closer to the group in the projection, claiming to be potential members. The user may have missed them when making the selection. This projection can be used to regain the missing parts. The Separate projection exhibits differences between the group and the other data. If there are boundary points, they will stand out in the projection. It facilitate the study of cluster boundaries. The above benefits largely owe to the combination of local optimization and focus + context technique. In previous works with only local projections~\cite{DBLP:journals/tvcg/YuanRWG13}, it's hard to modify a focus without any context information.

We support the modification by providing two other brushing modes. In the 'Increase' mode, whatever chosen by the user will be added into the focus group. In the 'Decrease' mode, the user can choose among group members without being affected by the other data. After the modification is made, all featured projections will be updated. Smooth transitions are applied during the update. We keep an orthogonal mapping in each frame of the transition~\cite{cook2004computational}, in order to help maintain an intact mental model of the data space. The user can continue the analysis and modification, until he gets a satisfying result.

\subsection{Focus Comparison in the Projection Map}
During the exploration, there will be times when the user needs to store the result. For example, when sub-clusters are found, the large cluster should be stored before the exploration goes into details. Besides, it's necessary to compare different focuses regarding their features. For these purpose, we provide the focus list, along with a map of all featured projections.

In the focus list, users can store the current focus or retrieve it at any time. Each focus is represented as a node (Figure~\ref{fig:map}). Its size denotes the data size. Users can name a focus, or assign it some color. Specially, there is a fixed node called 'All Data'. Applying Expand metric to it will get the global PCA projection.

For every focus in the list, its featured projections are shown as glyphs in the projection map. Different glyphs represent different types of projections, as shown in Figure~\ref{fig:map}(b). Dimension weights are also displayed as small histograms to help compare the projections. Clicking on a glyph can retrieve the focus and the corresponding projection. To quantify distances between any two projection, we refer to the manifold learning domain. It has been proved that any two 2D projections lie on the same manifold, which is called the Grassmann manifold. We measure projection distances by geodesic distances on the manifold~\cite{absil2004riemannian}. The map is constructed based on the distance matrix using MDS.

In the projection map, users can compare features of different focuses. For example, two focus groups may have the same reasons for the grouping (i.e. intra-group similarities), while having different inner-group diversities (see Figure~\ref{fig:map}). It helps to understand different local data in the context of featured dimensions. In addition, users can plan their own high-dimensional tours in this map. A similar idea has been proposed in the TripAdvisor~\cite{DBLP:journals/tvcg/NamM13}, as an extension of the Grand Tour~\cite{asimov1985grand}~\cite{cook1995grand}. But their projections are driven by dimensions without explicit semantics. In comparison, each spot in our map is tightly related to some local data and a certain relationship. It makes the tour more targeted and easier to interpret.
\section{Case Study}
In this section, we demonstrate the effective of our method with two real-world datasets.

\subsection{USDA Food Data}
The first case comes from the USDA Food Composition Data Set (http://www.ars.usda.gov/).  The dataset describes nutrients of a collection of raw or processed foods. It has been used in some previous works~\cite{DBLP:conf/ieeevast/TatuMFBSSK12}~\cite{DBLP:journals/tvcg/YuanRWG13} for their case studies. After preprocessing, the dataset contains 722 records and 18 dimensions.

\label{section:casestudy}
\section{Discussion}
\label{section:discussion}
\col{Outline for discussion:
\begin{enumerate}[(1)]
\item Extension: other featured projections, non-linear projections
\item Impact of data scale: increased dimensionality or data size, try a larger dataset with higher dimensionality
\item Comparison: local-preserving algorithms, projections for classification
\item Problems: lack of navigation, overlapping dimensions in a projection
\end{enumerate}
}
\section{Conclusion}
In this paper, we propose an interactive method, that helps customize a linear projection to facilitate the analysis of user-defined local data. We also provide various kinds of techniques to support a fluent high-dimensional data exploration. Users are assisted to discover, analyze, modify and compare multiple pieces of local data.

%% if specified like this the section will be committed in review mode
\acknowledgements{
The authors wish to thank A, B, C. This work was supported in part by
a grant from XYZ.}

\bibliographystyle{abbrv}
%%use following if all content of bibtex file should be shown
%\nocite{*}
\bibliography{viewpointchanger}
\end{document}
