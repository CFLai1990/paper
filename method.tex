\section{High-dimensional Data Exploration Guided by Distortion-Free Local Analysis}
As shown in the workflow, the proposed method supports a four-step exploration. In this section, we'll elaborate details of our method in each step of the exploration process.
\label{section:method}
\subsection{Discovering Interesting Local Data}
Following Shneiderman's suggestion~\cite{DBLP:conf/vl/Shneiderman96}, we first provide the PCA projection as an overview of the data. Then we help the user find an interesting subset as the focus of subsequent local analysis.

In a projection, there are two situations where some local data is considered interesting. The first case is related to distance distortion. Incorrect distances result in false neighborhood relationships. Closely distributed data may be far away in the original space and vice versa. Data involved in a distorted local area is regarded informative in the projection. It's also the basic idea in previous works concerning about local errors~\cite{DBLP:journals/cg/MartinsCMT14}~\cite{DBLP:journals/tvcg/StahnkeDMT16}. But such analysis can only focus on each datum at a time. It's hard to describe group relationships in this context. In the second case, the data is involved in some featured relationships, like being an outlier or a cluster. The relationship may have been weakened, but it's still strong enough to appear in the current projection. Hence, it makes a reasonable focus for a further study. Besides, it's suitable to describe a group of data in the context of relationships, rather than distance errors.

To put it simply, distortion analysis focuses more on the neighborhood of a single datum. Relationship analysis helps to study a group of data. Regarding the two cases, we adopt different means to help the user find an interesting local focus.

\subsubsection{Datum Suggestion Based on Distance Distortion}
For any given projection, we consider a datum interesting if its distances to other data have been severely distorted. To measure the distortion, we accumulate distance errors for each datum in the projection:
$$Error(\mathbf{x}_{i}^{\prime}) = \sum\limits_{j=1}^{n}(Dist(\mathbf{x}_{i}, \mathbf{x}_{j}) - Dist(\mathbf{x}_{i}^{\prime}, \mathbf{x}_{j}^{\prime}))^{2}, i = 1,2,\cdots n$$
Here the $\mathbf{x}_{i}$ and $\mathbf{x}_{i}^{\prime}$ represents the original data and the projected data respectively. Distance is measured by the Euclidean distance metric, taking into account all dimensions. We use point size to encode the accumulated distortion of each datum, as shown in Figure. \col{(Figure to be added.)}

On the other hand, we provide interactive hints to reveal the real distances. The approach is similar to that used in~\cite{DBLP:journals/tvcg/StahnkeDMT16}, but uses a different metaphor. When user hovers on the projection, we construct a so-called 'high-dimensional lantern' using interpolation. Assume that the hovered position corresponds to a two-dimensional datum $\mathbf{p}^{\prime}$, we interpolate its high-dimensional counterpart as follows:
$$\mathbf{p} = \sum\limits_{i=1}^{n}\mathbf{w}_{i}\cdot\mathbf{x}_{i} =  \sum\limits_{i=1}^{n}\frac{Dist(\mathbf{x}_{i}^{\prime}, \mathbf{p}^{\prime})^{-1}}{\sum\limits_{j=1}^{n}Dist(\mathbf{x}_{j}^{\prime}, \mathbf{p}^{\prime})^{-1}}\cdot\mathbf{x}_{i}$$
The interpolation weight $\mathbf{w}_{i}$ of data $\mathbf{x}_{i}$ depends on its distance to the hovered place in the projection. Closer data get larger weights. When user hovers right on $\mathbf{x}_{i}^{\prime}$, $\mathbf{w}_{i}$ equals $1$ while all the other weights get $0$. The result equals the original data: $\mathbf{p} = \mathbf{x}_{i}$.

By the interpolation, we aims to infer what kind of data is desired by the user. Then this desired point acts as a high-dimensional lantern, shedding lights on all the other data to indicate their distances. With the lighting metaphor, we encode distance information using the saturation tunnel in HSL color space:
$$Saturation(\mathbf{x}_{i}^{\prime}) = \max{\{(\alpha d_{i}^{2} + \beta d_{i} + \gamma)^{-1}, 1\}},$$
$$d_{i} = Dist(\mathbf{x}_{i}, \mathbf{p}), \quad i = 1,2,\cdots n$$
The data gets high saturation, if it's close to the interpolated point in the original space. The parameters $\alpha,\ \beta$ and $\gamma$ come from the inverse-square law of the lighting model. Empirical values are chosen to accommodate most datasets. When some datum has a large distance distortion, its lights will not be able to illuminate its neighbors in the projection. In contrast, some far away points will be highlighted as the real neighbors. Figure. demonstrates the actual effect. \col{(Figure to be added.)}

In summary, large data points are potentially interesting data with high distortion and inconsistent illumination. With the hints, user hovers around the projection like experiencing an adventure. He holds a lantern to explore unknown structures in the complex data space. Compared to~\cite{DBLP:journals/tvcg/StahnkeDMT16}, this method enables a more smooth and natural perception of distance information.

\subsubsection{Cluster Suggestion Based on Projected Relationships}
Automatic clustering algorithms play an important role in previous works~\cite{DBLP:conf/ieeevast/NamHMZI07}~\cite{DBLP:journals/cgf/LeeKCSP12}~\cite{DBLP:journals/cgf/LiuWTBP15}. Users are either given the clustering results, or assisted in toning parameters of the algorithm. However, it's not intuitive to drive the clustering by parameters, since the algorithm is often a black box to users. Besides, it's hard for users to understand causes and details about the clusters, let alone modifying them or discovering new ones.

In our method, we decide not to provide global clustering results. Instead, we suggest an interesting group of data by examining projection clusters. It is based on the fact that, no additional or prior knowledge should be assumed in a free exploration. Users choose their focuses based on what they perceive. We only help to reveal real structures of the chosen focus. Users can still take full control of the clustering process, after they get the local insights.

Basically, lots of clustering algorithms can be applied to identify projection clusters~\cite{DBLP:conf/ieeevast/Kandogan12}. We adopt a variant of DBSCAN~\cite{zhou2012research} whose parameters are adaptive to the data. We choose DBSCAN because it can efficiently identify clusters in any shape. The self-adaptive parameters make it applicable to most datasets without the need of manual toning. Refer to Figure. for the effect of cluster suggestion. \col{(Figure. to be added.)} Users can choose a suggested cluster by simply clicking it. The suggestion only clarifies dominant relationships perceived by the user. It doesn't provide any extra information beyond the projection. If the user doesn't feel satisfied with the suggestion, he can choose his own focus by brushing the data.

\subsection{Pursuing Featured Projections for the Focus}
We call a chosen datum the focus point, and call a chosen group the focus group. After some focus is chosen, we generate projections to enhance it local relationships. In our method, we define three types of featured relationships that are most concerned in the analysis. Accordingly, three projections are provided to enhance these local relationships.

\subsubsection{Featured Local Relationships}
Relationships are defined based on distances. We can take a look at the distance matrix, to examine what kind of relationships should be taken into consideration in the local analysis. Given a focus group, the distance matrix of all data is divided into three parts (see Figure. ). \col{(Figure to be added.)} The first part describes distances between group members. The second part is about distances between the group and the other data. The last part describes distances among the other data. Since the last part has nothing to do with the focus, we simply ignore it. For the remaining parts, we consider the chosen data to be either 'similar' or 'dissimilar'.

By revealing the similarities among group members, users can learn about in which aspects they are most similar. It helps to comprehend why these data gather into a cluster in the projection. Enhancing dissimilarities, on the other hand, tells about the major differences among group members. Moreover, if there are sub-clusters within the group, the differences among them will be more prominent. In other words, it can reveal hidden local relationships. The similarities between the group and the others is not of interest, as far as we are concerned. In contrast, by enhancing the dissimilarities, we can show why this group is different from the others. The idea resembles that of Linear Discriminant Analysis (LDA), expect that we did not regard the other data as a same class. In summary, three types of relationships are found most informative in the local analysis.\note{We name them 'intra-focus similarity', 'intra-focus dissimilarity' and 'inter-focus dissimilarity' respectively.}

The situation is alike with a focus point, but no more groups exist here. As mentioned before, the analysis of a single datum concerns about its neighborhood. Hence, we replace the concept of group with neighborhood, and explain the three relationships accordingly.

\subsubsection{Relationship Enhanced Projections}
\subsubsection{Subspace Suggestion}
\subsection{From Focus to Cluster}
\subsubsection{Focus Improvement}
\subsubsection{Cluster Comparison via Viewpoint Map}
