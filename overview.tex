\section{Overview}
\label{section:overview}

\note{
\col{Outline for overview:
\begin{enumerate}[(1)]
\item Reducing local distortion
\item Enhancing local relationships
\item Workflow
\end{enumerate}
}

In this work, we propose local featured projections, to facilitate the analyses of user-defined focuses, to facilitate high-dimensional local data analysis. These projections

In this work, we propose to steer high-dimensional data exploration via local analyses in linear projections. To facilitate the analysis, we help to reduce local distortions, and enhance local relationships. In this section, we first clarify the concept of distortion reduction and relationship enhancement. Then we give an overview of the interactive exploration process supported by the locally optimized projections.

\subsection{Local Distortion Reduction}
Distortion usually refers to the gap between original data distances and the projected distances. We call it the \emph{distance distortion}. Our approach helps to reduce such distortions locally for a more faith perception. However, it may not be enough to support data feature analyses. There have been lots of dimension reduction techniques~\cite{jolliffe2002principal}~\cite{borg2005modern}~\cite{tenenbaum2000global}~\cite{roweis2000nonlinear}, aiming to reduce distance distortions globally. But a more recent research~\cite{DBLP:journals/tvcg/EtemadpourMPMOL15} has shown that, those projections cannot guarantee a good performance for certain analytic tasks. The main cause, in our opinion, is the existence of relationship distortions.

By 'relationship', we refer to a relative concept of distance, i.e. 'close' or 'far'. In the high-dimensional space, relationship should be defined in a certain data range and dimensional subspace. Assume that there are four data items distributed in a two-dimensional plane, as shown in Figure. \col{(Figure to be added.)} When talking about data A, B and C, we can say that C is far away from B (compared to A). But when talking about data B, C and D, C seems close to B (compared to D). On the other hand, C is closer to B than A in dimension X, while the opposite happens in dimension Y. When combing all data and dimensions, weaker relationships give way to the stronger ones. The weak ones (e.g. C is closer to B than A) can no longer be perceived. Even the strong ones are not as obvious as in the original context. We call it the \emph{relationship distortion}. The situation is alike in more complex real-world datasets. Integrated measurements cannot reflect local relationships precisely. That's why approximating the overall distances cannot guarantee enhanced data features.

In contrast, reducing distortion of the interested relationship can support a more targeted analysis. For example, when we talk about two objects being similar, we tend to ignore their differences temporally and vice versa. We believe that distortion reduction should be built upon certain analytic targets, namely the local data and relationships. It is the key to revealing hidden local features. That's the reason we provide two different types of distortion reduction, for different purposes. In fact, we will prove that distance distortion is a special case of relationship distortion. Both goals can actually be achieved in the same framework.\note{ Specifically, we allow users to focus on a data subset to accommodate relationships in different ranges. In addition, projection pursuit is applied to enhance relationships in different subspaces.} More details will be introduced in Section~\ref{section:method}.

\subsection{Workflow}
}

In this work, we aim to facilitate local data analysis in linear projections. We propose an interactive and exploratory scheme to help users discover, analyze, modify, and compare different POI local data. To be specific, our method supports a four-step data exploration (Figure~\ref{fig:workflow}):
\note{
Our method promotes an efficient high-dimensional data exploration, by facilitating distortion-free local analysis. It not only helps to maintain a faithful perception, but aims to enhance local features for better information mining. To be specific, the exploration contains four steps (see Figure~\ref{fig:workflow}):
}
\begin{enumerate}[Step. 1:]
 \item \textbf{Focus Search}: First, we present a global projection as an overview (Figure~\ref{fig:workflow}(a)) of the data. Users can choose any data subset in the layout and name it as the POI, which is also called a \textbf{focus}. We define two types of focuses, i.e. the focus point and the focus group, regarding whether the POI includes multiple samples. We also make recommendations for both types. Users can simply follow our suggestions if they don't know what to choose.
\note{
 In the given projection, we help the user find an interesting piece of local data (Figure~\ref{fig:workflow}(b)). We assume a datum interesting if its distances to the others has been distorted. We consider a group of data interesting if they appear as a cluster. Hence, we display distance distortions and projection clusters respectively, to suggest an interesting datum or group. The data chosen by the user, either a point or a group, is called a focus.}
 \item \textbf{Enhancing Features of the Focus}: After some focus is chosen, we customize a linear projection to enhance features of the focus. By features, we refer to three kinds of featured relationships we find most informative in the local analysis (Figure~\ref{fig:workflow}(d)). Different features serve different analytic tasks. Besides the enhancement, we further reveal dimensions that may relate to the feature, and thus suggest a feature-prominent subspace. It helps to interpret dimensional causes of the underlying data relationships.
 \item \textbf{Modifying the Focus}: When analyses are done in the projection, the user may need to modify the current focus or simply change his interest. We provide various means to allow different interactions with the focus. User can choose a new focus, or add / remove samples from the current POI without being interfered by irrelevant data. Projections update along with the focus, allowing the user to seamlessly change his focus.
 \note{
 When the focus is a data group chosen from the projection, it could be a false cluster or missing some important pieces. The focus-enhanced projections (Figure~\ref{fig:workflow}(e)) provide insights from multiple aspects. 
 They may reveal truths that could not be found in the previous projection, like sub-clusters, outliers, etc. We provide means to help shape the focus into a more consistent and complete group. Whenever the focus is updated, the focus-enhanced projections will also be updated. The process returns to Step~2. This loop goes on until the user fully understands the data in focus, or acquires a featured cluster. After that, he can save the results and transfer his attention to another focus, returning to Step~1.}
 \item \textbf{Focus Comparison}: When a valuable POI is found, the user can store it in the Focus List for further analysis (Figure~\ref{fig:workflow}(f)). We also provide the Projection Map (Figure 1(e)) that shows featured projections as glyphs for all focuses in the list. It helps to compare different focuses, and navigate the high-dimensional exploration.
\end{enumerate}
In this four-step exploration, users can pick up any local data and feature, and get both structural and dimensional insights from the locally enhanced projections. The analysis no longer suffers from invisible and uncontrollable distortions, since the feature to be analyzed is already enhanced to the greatest extent. Moreover, users are now able to handle and compare multiple focuses, and retrieve the explored ones at any time.

We develop a prototype system called FocusChanger (Figure~\ref{fig:teaser}) to support this exploratory pipeline. It consists of five parts: Projection View, Information Panel, Control Panel, Focus List and Projection Map. We'll introduce the function of each part as we elaborate the technical details in the next section.

\note{
In the exploration process, users can transfer seamlessly between different local data. The projections can effectively preserve information of the focus, without considering the irrelevant parts. They not only show hidden local relationships, but disclose their causes in the context of dimensions.  In the next section, we'll introduce in detail how we support the above exploration in each step.
}