\firstsection{Introduction}
\maketitle
Dimension-reduced projection is widely used for high-dimensional data analysis. It seeks to approximate the original distribution in a low-dimensional space. Such approximation is often globally optimized to make a good overview of the data. But due to approximation errors, relationships among data will inevitably be distorted. The distortions are hard to ignore when it comes to local data. They largely harm the perception of local structures, yet are often transparent to users. Even if they are shown in the projection~\cite{DBLP:journals/tvcg/StahnkeDMT16}, users have no means to control their distribution. In general, dimension-reduced projections make good overviews, but are not suitable for local data analysis.
%change citation here.

Nevertheless, local analysis is not only necessary, but also an efficient way to explore high-dimensional data. On one hand, a sole overview cannot show all aspects.\note{especially when the dimensionality is high.} After a quick glance, users tend to pick up some local structure (e.g. clusters, outliers) and go into details. It helps them fully understand the data and discover more useful information. On the other hand, featured local data are good breakthrough points for an efficient exploration. Traditional methods~\cite{DBLP:journals/tvcg/NamM13}~\cite{DBLP:journals/cgf/JeongZFRC09}~\cite{DBLP:journals/tvcg/LehmannT13} allow users to manipulate projection dimensions. But such exploration is blind and exhausting, since the data space is huge while users have no clue where to go. Keeping some cluster However, such Lots of traditional methods provide clustering results as starting points.