\firstsection{Introduction}
\maketitle
\note{
\col{Outline for introduction (section to be modified):
\begin{enumerate}[(1)]
\item Necessity: It's necessary to conduct high-dimensional local data analyses, especially in a projection
\item Difficulties: Projection is not suitable for local analysis. Users concern about local structures, but distances are distorted while local relationships are weaken or hidden
\item Previous works and their defects: Compared with: local-reserved projections and machine learning, dimension-driven projection, multiple-view exploration.
\item Goal and contribution: Facilitate local analysis in a projection, by reducing distortions and enhancing local relationships
\end{enumerate}
}

(1) Defects of a projection: distortions of the neighborhood, weakened or hidden local relationships.
(2) Users could have focuses when seeing a projection. It's necessary to conduct a local analysis.
(3) Manifold learning: they seek to preserve the locality for each single point, but still too hard, and probably not necessary. Besides, they missed the dimensional context when using a non-linear mapping.
(4) Enhance a focus in the projection: it not only alleviates the burden of projections, but is more targeted and more efficient in using the display space. We only give users what they want. There should be a separation between the focus and the context. It's always difficult to strike a balance among all data.
}
Dimension-reduced projection is widely used for high-dimensional data analysis. It approximates a high-dimensional data distribution in a low-dimensional space. Such approximation is often made a global one, that improves the overall mapping quality by striking a balance among all data. Two well-known examples are the Principle Component Analysis (PCA) and the Multidimensional Scaling (MDS). They usually make good overviews of the data. But still, it's impossible to preserve all information without any loss. Projection distortion is a classic case of such loss, where inaccurate distance mapping leads to unfaithful interpretation of data relationships. The problem gets more sever when data size or dimensionality increases.

However, it's local analysis rather than the overview that suffers most from distortions. When facing with a projection, the user may find a point of interest (POI) with prior knowledge, or is simply attracted by an interesting local region. He may wish to observe the POI more precisely, while not so concerned about the other data. But the fact is, users are often not aware of distortions. They have no idea which part of the projection is more accurate, and easily get misguided. Even when distortions are informed~\cite{DBLP:journals/ijon/Aupetit07}~\cite{DBLP:journals/cgf/LiuWBP14}, there are no means for users to reduce them. You see, it's already hard to preserve one piece of local data, not to mention accommodating multiple POIs in a single projection. It's simply impossible for a sole overview to fulfill analytic tasks regarding various POIs.

Over the last few decades, lots of projections have been proposed to promote local data analyses. In machine learning field, researchers believe that most real-world data lie on a low-dimensional manifold. They seek to capture localities on the manifold, and restore them in the projection~\cite{DBLP:journals/neco/BelkinN03}~\cite{roweis2000nonlinear}~\cite{DBLP:journals/corr/cs-LG-0212008}. These methods are highly effective in preserving local neighborhoods, but they tend to ignore semantics of data dimensions. They can tell you which data are similar, but not 'in what aspects' they are similar. It makes them unsuitable for general data explorations, where dimensional analyses are quite important. In visualization field, we often answer the 'aspect' question by feeding data into another view (e.g. parallel coordinates) beyond the projection. But still, the task is not straightforward. Frequent view changing and linking could also hinder a smooth exploration. More delicate methods allow users to manually alter dimensions of the projection~\cite{DBLP:journals/cgf/JeongZFRC09}~\cite{DBLP:journals/tvcg/NamM13}~\cite{DBLP:journals/tvcg/LehmannT13}. However, users have to search blindly in the parameter space of dimension weights. It's difficult to find a satisfying projection to observe some certain POI.

Given all efforts made to promote local analysis, maybe we can reconsider the problem from a different perspective. As we know, it consumes both time and display space to accommodate all data in a projection. Yet a sole projection can hardly preserve both data similarities and dimensional insights, and still satisfy various local analyses. Once the mapping doesn't meet users' needs, all efforts will be wasted. But what if we learn the POI in advance from users? Perhaps we can use the resources more wisely. The projection can pay full attention to preserve features of the POI, instead of worrying about the unconcerned data. Projection quality will be improved when fewer data compete for the display space. With an improved mapping, a linear projection could be good enough to preserve data locality, allowing it to also maintain dimensional contexts. At last, we only need to make a new projection once the user changes his POI or analytic tasks. All problems seem settled in this way. It inspires us to develop the method we are going to present in this paper.
\note{Yuan et al.~\cite{DBLP:journals/tvcg/YuanRWG13} have taken a similar strategy to organize subspace exploration. In their work, users separate a data subset from the majority and project it in a chosen subspace. The method we would like to propose here, also falls into the same category. However, we focus more on local data analysis, rather than subspace mining.
}

In this paper, we propose an interactive scheme to customize a linear projection to facilitate analysis of user-defined local data. Specifically, we allow user to define a focus in a projection. Then we offer multiple ways to alter the projection, to either reduce local distortions, or reveal hidden features of the focus. The other data are displayed as context information. Based on the localized projections, we further reveal subspaces where the focus is most featured. It helps to interpret the focus in the context of dimensions. In addition, we provide various means to support the data exploration at different stages. Users are assisted to discover, analyze, modify and compare different focuses. Compared to global projections, our method is more adaptive to the ever-changing local analytic tasks. Users are able to transfer seamlessly between different POIs, while supplied with high-quality and informative projections.

In summary, our contributions include:
\begin{itemize}
\item Given the user-defined POI data, we provide linear projections with enhanced local features to support various kinds of local analytic tasks.
\item Our method supports an interactive high-dimensional data exploration, where users are assisted to discover, analyze, modify and compare interesting pieces of local data.
\end{itemize}

The remainder of this paper is structured as follows. In the next section, we briefly review the related literature. Section~\ref{section:overview} gives an overview on the proposed method based on the exploration process it supports. Then we elaborate each part of the method in detail in Section~\ref{section:method}. Section~\ref{section:casestudy} presents case studies to demonstrate the effectiveness of our method. In session~\ref{section:discussion}, we discuss about weaknesses and potential improvements. At last, we end this paper with the conclusions.

\note{

What if we know the POI in advance, and use the resources more efficiently?

The other data are kept as the context. That's where interactive techniques come in. in this work, we enable users to

Yuan et al.~\cite{DBLP:journals/tvcg/YuanRWG13} let users specify their interested data subset. The chosen subset is then isolated from the other data, and displayed in a separate MDS. The local projection serves well for analyzing the POI. But distortions and local relationships are not studied thoroughly.

Yuan et al.~\cite{DBLP:journals/tvcg/YuanRWG13} proposed a creative solution that promotes subspace explorations. Users can separate the POI data and project them in a dimensional subspace.


In fact, no global projection can satisfy the ever-changing local analysis requirements. Efforts made to generate the projection are wasted, once the results cannot meet users' needs. On the other hand, it consumes both time and the display space to accommodate all data in a projection. If we know the POI in advance, we can use these resources more efficiently, by paying more attention to preserve the POI information.

In that circumstance, it's a reasonable choice to enhance POI information, even at the cost of the others. But information loss like the distortions are often transparent to users. Even they are displayed~\cite{DBLP:journals/ijon/Aupetit07}~\cite{DBLP:journals/cgf/LiuWBP14}, there is no means for users to control the distortions.

As suggested by Shneiderman in his information seeking mantra~\cite{DBLP:conf/vl/Shneiderman96}, detail analysis is  Besides,
Moreover, such distortions are usually transparent to users. unpredictable
projection distortions are inevitable. Distances shown in the projection are probably inaccurate, leading to an unfaithful interpretation of data relationships.

Such approximation is often globally optimized to make a good overview of the data. But a sole overview cannot show all aspects. Detail analysis is a necessary supplement, as suggested by Shneiderman in his information seeking mantra~\cite{DBLP:conf/vl/Shneiderman96}. However, due to approximation errors, data relationships will inevitably be distorted. The distortions are hard to ignore when it comes to local data. They largely harm the perception of local structures, yet are often transparent to users. Even if distortions are shown in the projection~\cite{DBLP:journals/tvcg/StahnkeDMT16}~\cite{DBLP:journals/ijon/Aupetit07}, users have no means to control their distribution. In general, globally optimized projections make good overviews, but are not suitable for local data analysis.
%change citation here.

One way to alleviate this problem, is to observe the data in a different perspective. Some previous works~\cite{DBLP:journals/cgf/JeongZFRC09}~\cite{DBLP:journals/tvcg/NamM13}~\cite{DBLP:journals/tvcg/LehmannT13} allow users to change the projection by adjusting dimension weights. It helps to clarify detailed structures and find out more useful information. To help build a targeted analysis, featured clusters are often provided beforehand~\cite{DBLP:journals/tvcg/NamM13}~\cite{DBLP:journals/cgf/LiuWTBP15}. But users know nothing about the given clusters. They have to solve their puzzles by manually searching the enormous data space. It's a blinded and exhausting process, where users have no clue how to steer the dimensions to get a better view. Even if interesting projections are found, it's hard to explain or assess them without distortion information. There is no guarantee that local structures will be shown more precisely in the new perspective.

The low-efficiency of dimension-based exploration is caused by three facts. First, dimension weights are too abstract for users to understand and control. It's hard to explain something like $'25\% Weight + 75\% Height'$. Second, users don't know about the interplay between dimension weights and the projection. They cannot foresee the effects of changing weights, which directly leads to a try-and-error process. Third, the exploration is blinded without a clear target. It somehow depends on luck to get informative findings.

Compared to dimensions, it's more reasonable and more efficient to explore projections via data features. It is the core spirit behind feature-based mining techniques. Projection pursuit~\cite{DBLP:journals/tc/FriedmanT74} searches for projections to optimize predefined indices. The rank-by-feature framework~\cite{DBLP:journals/ivs/SeoS05} ranks a series of projections by their interestingness. In more recent works, users are involved to specify desired features~\cite{DBLP:journals/tvcg/JohanssonJ09} and relationships~\cite{DBLP:journals/tvcg/HuBMHNL13}~\cite{DBLP:journals/tvcg/Gleicher13}. These methods, though effective, largely depend on predefined metrics or users' prior knowledge. It makes them unsuitable for an interactive exploration starting from scratch. In fact, such techniques have the power to reduce local distortions in a linear projection. But little attention had been paid in this aspect.

Inspired by previous works, we wonder if we could introduce projection pursuit into interactive explorations, to solve the local distortion problem. Users can specify their point of interest (POI) data. Then we return projections with the least local distortion for a better analysis. There are good reasons for such kind of exploration. First, data relationships are easier to perceive, understand, and control. Users may not be familiar with dimensions, but they are interested in, and also familiar with featured relationships like clusters and outliers. They can decide which part of data should be studied in detail. Second, projection pursuit can be used to reduce distortion of local data. It's far more efficient than manual controls. Users can be free from parameter tuning, and focus more on data features. Third, the exploration is more targeted with a POI. Besides, users don't have to explain projections by their dimensions. It makes more sense to explain them in the context of data features.

These thoughts were pushed forward into the method we'd like to present in this paper. It is an interactive approach to steer high-dimensional data exploration, through consecutive local analyses in locally optimized projections. To be specific, the exploration contains four steps. First, for any given projection, we help the user find a piece of interesting local data. Then for some chosen data, we provide distortion-free projections to enhance its features for a better perception. In addition, subspaces related to the projections can reveal causes of the features. Since the chosen data could be a false cluster or missing some important pieces, we also help to shape it into a more consistent and complete cluster. At last, whenever some valuable local data is found, the user can store it for further analyses. We provide a 'projection map' containing all featured projections to support the analysis. It helps to compare different pieces of data, and organize the high-dimensional exploration. More details will be introduced in the following sections.

In summary, our contributions include:
\begin{enumerate}[(1)]
\item We help users customize dimension-reduced projections for a targeted and distortion-free local data analysis.
\item We propose a data-based interactive exploration method. Users are able to steer the exploration efficiently, by focusing on local data analysis, rather than dimension tuning.
\end{enumerate}

Nevertheless, local analysis is not only necessary, but also an efficient way to explore high-dimensional data. On one hand, a sole overview cannot show all aspects.\note{especially when the dimensionality is high.} After a quick glance, users tend to pick up some local structure (e.g. clusters, outliers) and go into details. It helps them fully understand the data and discover more useful information. On the other hand, featured local data are good breakthrough points for an efficient exploration. Traditional methods~\cite{DBLP:journals/tvcg/NamM13}~\cite{DBLP:journals/tvcg/LehmannT13} allow users to change the projection by adjusting dimensions. But such exploration is blind and exhausting, since the data space is huge while users have no clue where to go. Compared to dimensions, data features are easier to perceive and explain. That's why dimension-based explorations often provide featured clusters as start points~\cite{DBLP:journals/tvcg/NamM13}~\cite{DBLP:journals/cgf/LiuWTBP15}. It's also the spirit behind feature-based projection mining techniques, such as projection pursuit~\cite{DBLP:journals/tc/FriedmanT74} and the rank-by-feature framework~\cite{DBLP:journals/ivs/SeoS05}.
}