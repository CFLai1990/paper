\firstsection{Introduction}
\maketitle
Dimension-reduced projection is widely used for high-dimensional data analysis. It seeks to approximate the original distribution in a low-dimensional space. Such approximation is often globally optimized to make a good overview of the data. But a sole overview cannot show all aspects. Detail analysis is a necessary supplement, as suggested by Shneiderman in his information seeking mantra~\cite{DBLP:conf/vl/Shneiderman96}. However, due to approximation errors, data relationships will inevitably be distorted. The distortions are hard to ignore when it comes to local data. They largely harm the perception of local structures, yet are often transparent to users. Even if distortions are shown in the projection~\cite{DBLP:journals/tvcg/StahnkeDMT16}~\cite{DBLP:journals/ijon/Aupetit07}, users have no means to control their distribution. In general, globally optimized projections make good overviews, but are not suitable for local data analysis.
%change citation here.

One way to alleviate this problem, is to observe the data in a different perspective. Some previous works~\cite{DBLP:journals/cgf/JeongZFRC09}~\cite{DBLP:journals/tvcg/NamM13}~\cite{DBLP:journals/tvcg/LehmannT13} allow users to change the projection by adjusting dimension weights. It helps to clarify detailed structures and find out more useful information. To help build a targeted analysis, featured clusters are often provided beforehand~\cite{DBLP:journals/tvcg/NamM13}~\cite{DBLP:journals/cgf/LiuWTBP15}. But users know nothing about the given clusters. They have to solve their puzzles by manually searching the enormous data space. It's a blinded and exhausting process, where users have no clue how to steer the dimensions to get a better view. Even if interesting projections are found, it's hard to explain or assess them without distortion information. There is no guarantee that local structures will be shown more precisely in the new perspective.

The low-efficiency of dimension-based exploration is caused by three facts. First, dimension weights are too abstract for users to understand and control. It's hard to explain something like $'25\% Weight + 75\% Height'$. Second, users don't know about the interplay between dimension weights and the projection. They cannot foresee the effects of changing weights, which directly leads to a try-and-error process. Third, the exploration is blinded without a clear target. It somehow depends on luck to get informative findings.

Compared to dimensions, it's more reasonable and more efficient to explore projections via data features. It is the core spirit behind feature-based mining techniques. Projection pursuit~\cite{DBLP:journals/tc/FriedmanT74} searches for projections to optimize predefined indices. The rank-by-feature framework~\cite{DBLP:journals/ivs/SeoS05} ranks a series of projections by their interestingness. In more recent works, users are involved to specify desired features~\cite{DBLP:journals/tvcg/JohanssonJ09} and relationships~\cite{DBLP:journals/tvcg/HuBMHNL13}~\cite{DBLP:journals/tvcg/Gleicher13}. These methods, though effective, largely depend on predefined metrics or users' prior knowledge. It makes them unsuitable for an interactive exploration starting from scratch. In fact, such techniques have the power to reduce local distortions in a linear projection. But little attention had been paid in this aspect.

Inspired by previous works, we wonder if we could introduce projection pursuit into interactive explorations, to solve the local distortion problem. Users can specify their point of interest (POI) data. Then we return projections with the least local distortion for a better analysis. There are good reasons for such kind of exploration. First, data relationships are easier to perceive, understand, and control. Users may not be familiar with dimensions, but they are interested in, and also familiar with featured relationships like clusters and outliers. They can decide which part of data should be studied in detail. Second, projection pursuit can be used to reduce distortion of local data. It's far more efficient than manual controls. Users can be free from parameter toning, and focus more on data features. Third, the exploration is more targeted with a POI. Besides, users don't have to explain projections by their dimensions. It makes more sense to explain them in the context of data features.

These thoughts were pushed forward into the method we'd like to present in this paper. It is an interactive approach to steer high-dimensional data exploration, through consecutive local analyses in locally optimized projections. To be specific, the exploration contains four steps. First, for any given projection, we help the user find a piece of interesting local data. Then for some chosen data, we provide distortion-free projections to enhance its features for a better perception. In addition, subspaces related to the projections can reveal causes of the features. Since the chosen data could be a false cluster or missing some important pieces, we also help to shape it into a more consistent and complete cluster. At last, whenever some valuable local data is found, the user can store it for further analyses. We provide a 'projection map' containing all featured projections to support the analysis. It helps to compare different pieces of data, and organize the high-dimensional exploration. More details will be introduced in the following sections.

In summary, our contributions include:
\begin{enumerate}[(1)]
\item We help users customize dimension-reduced projections for a targeted and distortion-free local data analysis.
\item We propose a data-based interactive exploration method. Users are able to steer the exploration efficiently, by focusing on local data analysis, rather than dimension toning.
\end{enumerate}

The remainder of this paper is structured as follows. In the next section, we briefly review the related literature. Section~\ref{section:overview} gives an overview on the proposed method based on the exploration process. Then we elaborate each part of our method in detail in Section~\ref{section:method}. Section~\ref{section:casestudy} presents case studies to demonstrate the effectiveness of our method. In session~\ref{section:discussion}, we have a discussion about weaknesses and potential improvements of our method. At last, we end this paper with the conclusions.

\note{
Nevertheless, local analysis is not only necessary, but also an efficient way to explore high-dimensional data. On one hand, a sole overview cannot show all aspects.\note{especially when the dimensionality is high.} After a quick glance, users tend to pick up some local structure (e.g. clusters, outliers) and go into details. It helps them fully understand the data and discover more useful information. On the other hand, featured local data are good breakthrough points for an efficient exploration. Traditional methods~\cite{DBLP:journals/tvcg/NamM13}~\cite{DBLP:journals/tvcg/LehmannT13} allow users to change the projection by adjusting dimensions. But such exploration is blind and exhausting, since the data space is huge while users have no clue where to go. Compared to dimensions, data features are easier to perceive and explain. That's why dimension-based explorations often provide featured clusters as start points~\cite{DBLP:journals/tvcg/NamM13}~\cite{DBLP:journals/cgf/LiuWTBP15}. It's also the spirit behind feature-based projection mining techniques, such as projection pursuit~\cite{DBLP:journals/tc/FriedmanT74} and the rank-by-feature framework~\cite{DBLP:journals/ivs/SeoS05}.
}