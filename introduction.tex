\firstsection{Introduction}
\maketitle
Dimension-reduced projection is widely used for high-dimensional data analysis. It seeks to approximate the original distribution in a low-dimensional space. Such approximation is often globally optimized to make a good overview of the data. But due to approximation errors, data relationships will inevitably be distorted. The distortions are hard to ignore when it comes to local data. They largely harm the perception of local structures, yet are often transparent to users. Even if distortions are shown in the projection~\cite{DBLP:journals/tvcg/StahnkeDMT16}~\cite{DBLP:journals/ijon/Aupetit07}, users have no means to control their distribution. In general, globally optimized projections make good overviews, but are not suitable for local data analysis.
%change citation here.

One way to alleviate this problem, is to observe the data in a different perspective. Some previous works~\cite{DBLP:journals/cgf/JeongZFRC09}~\cite{DBLP:journals/tvcg/NamM13}~\cite{DBLP:journals/tvcg/LehmannT13} allow users to change the projection by adjusting dimension weights. It helps to clarify detailed structures and find out more useful information. To help build a targeted analysis, featured clusters are often provided beforehand~\cite{DBLP:journals/tvcg/NamM13}~\cite{DBLP:journals/cgf/LiuWTBP15}. However, users know nothing about the given clusters. They have to solve their puzzles by manually searching the enormous data space. It's a blinded and exhausting process, where users have no clue how to steer the dimensions to get a better view. Even if interesting projections are found, it's hard to explain or assess them without distortion information.

Compared to dimension-based exploration, feature-based mining techniques are more efficient in revealing informative projections. Projection pursuit~\cite{DBLP:journals/tc/FriedmanT74} searches for projections to optimize predefined indices. The rank-by-feature framework~\cite{DBLP:journals/ivs/SeoS05} ranks a series of projections by their interestingness. In more recent works, users are involved to specify desired features~\cite{DBLP:journals/tvcg/JohanssonJ09} and relationships~\cite{DBLP:journals/tvcg/HuBMHNL13}~\cite{DBLP:journals/tvcg/Gleicher13}. These methods, though effective, largely depend on predefined metrics or users' prior knowledge. It makes them unsuitable for an interactive exploration starting from scratch. Besides, little attention had been paid to facilitate local data analysis.

In this work, we propose an interactive exploration method, that promotes consecutive local data analyses in locally optimized projections.

\note{
Nevertheless, local analysis is not only necessary, but also an efficient way to explore high-dimensional data. On one hand, a sole overview cannot show all aspects.\note{especially when the dimensionality is high.} After a quick glance, users tend to pick up some local structure (e.g. clusters, outliers) and go into details. It helps them fully understand the data and discover more useful information. On the other hand, featured local data are good breakthrough points for an efficient exploration. Traditional methods~\cite{DBLP:journals/tvcg/NamM13}~\cite{DBLP:journals/tvcg/LehmannT13} allow users to change the projection by adjusting dimensions. But such exploration is blind and exhausting, since the data space is huge while users have no clue where to go. Compared to dimensions, data features are easier to perceive and explain. That's why dimension-based explorations often provide featured clusters as start points~\cite{DBLP:journals/tvcg/NamM13}~\cite{DBLP:journals/cgf/LiuWTBP15}. It's also the spirit behind feature-based projection mining techniques, such as projection pursuit~\cite{DBLP:journals/tc/FriedmanT74} and the rank-by-feature framework~\cite{DBLP:journals/ivs/SeoS05}.
}